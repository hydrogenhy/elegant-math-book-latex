%%%%%%%%%%%%%%%%%%%%%%%%%%%%%%%%%%%%%%%%%
%  My documentation report
%  Objetive: Explain what I did and how, so someone can continue with the investigation
%
% Important note:
% Chapter heading images should have a 2:1 width:height ratio,
% e.g. 920px width and 460px height.
%
%%%%%%%%%%%%%%%%%%%%%%%%%%%%%%%%%%%%%%%%%


%----------------------------------------------------------------------------------------
%	PACKAGES AND OTHER DOCUMENT CONFIGURATIONS

%----------------------------------------------------------------------------------------

\documentclass[11pt,fleqn, openany]{book} % Default font size and left-justified equations
\usepackage{ctex}
\usepackage{url}
\usepackage[top=3cm,bottom=3cm,left=3.2cm,right=3.2cm,headsep=10pt,letterpaper]{geometry} % Page margins

\usepackage{xcolor} % Required for specifying colors by name
\definecolor{ocre}{RGB}{52,177,201} % Define the orange color used for highlighting throughout the book

% Font Settings
\usepackage{avant} % Use the Avantgarde font for headings
%\usepackage{times} % Use the Times font for headings
\usepackage{mathptmx} % Use the Adobe Times Roman as the default text font together with math symbols from the Sym­bol, Chancery and Com­puter Modern fonts
\usepackage{microtype} % Slightly tweak font spacing for aesthetics
\usepackage[utf8]{inputenc} % Required for including letters with accents
\usepackage[T1]{fontenc} % Use 8-bit encoding that has 256 glyphs
\usepackage{amsthm}

% Bibliography
\usepackage[style=alphabetic,sorting=nyt,sortcites=true,autopunct=true,babel=hyphen,hyperref=true,abbreviate=false,backref=true,backend=biber]{biblatex}
\addbibresource{bibliography.bib} % BibTeX bibliography file
\defbibheading{bibempty}{}

\input{structure} % Insert the commands.tex file which contains the majority of the structure behind the template

%----------------------------------------------------------------------------------------
%	Definitions of new commands
%----------------------------------------------------------------------------------------

\def\R{\mathbb{R}}
\newcommand{\cvx}{convex}
\begin{document}

%----------------------------------------------------------------------------------------
%	TITLE PAGE
%----------------------------------------------------------------------------------------

\begingroup
\thispagestyle{empty}
\AddToShipoutPicture*{\put(0,0){\includegraphics[scale=1.25]{esahubble}}} % Image background
\centering
\vspace*{5cm}
\par\normalfont\fontsize{35}{35}\sffamily\selectfont
\textbf{Book name}\\
{\LARGE book title}\par % Book title
\vspace*{1cm}
{\Huge maybe author name}\par % Author name
\endgroup

%----------------------------------------------------------------------------------------
%	COPYRIGHT PAGE
%----------------------------------------------------------------------------------------

\newpage
~\vfill
\thispagestyle{empty}

%\noindent Copyright \copyright\ 2014 Andrea Hidalgo\\ % Copyright notice

\noindent \textsc{Edited By xxx,balabalabala}\\

% \noindent \textsc{2023.7}\\ % URL

% \noindent 
如果很正式,需要去搜下这里要写什么。非正式场合在这里你可以写一些类似于前言的东西
 

\textit{mail:12345678@xxx.com}\\

\noindent \textit{July 2023} % Printing/edition date

%----------------------------------------------------------------------------------------
%	TABLE OF CONTENTS
%----------------------------------------------------------------------------------------

\chapterimage{head1.png} % Table of contents heading image

\pagestyle{empty} % No headers

\tableofcontents % Print the table of contents itself

%\cleardoublepage % Forces the first chapter to start on an odd page so it's on the right

\pagestyle{fancy} % Print headers again

%----------------------------------------------------------------------------------------
%	CHAPTER 1
%----------------------------------------------------------------------------------------

\chapterimage{head2.png} % Chapter heading image
\chapter{章节标题}
\section{一级标题}
这是正文这是正文这是正文这是正文这是正文这是正文这是正文这是正文这是正文这是正文。
\subsection{二级标题}
这是正文这是正文这是正文这是正文这是正文这是正文这是正文这是正文这是正文这是正文
\begin{definition}[模拟信号]
    时间连续\textbf{或}幅值连续的信号称为模拟信号。
\end{definition}

\begin{theorem}[冲激函数的筛选性质]
    这是一个定理
\end{theorem}

这是正文这是正文这是正文这是正文这是正文这是正文这是正文这是正文这是正文这是正文

\begin{example}
    判断下列信号是连续信号还是离散信号:\\
    1. $f(t)=sin(t)$\\
\end{example}

下面我要说若干点:
\begin{itemize}
    \item first one
    \item second one
\end{itemize}

\begin{center}
    \includegraphics[scale=0.5]{Pictures/1_1.png}
\end{center}

\subsubsection{三级标题}
试试链接:\url{http://bilibili.com}






%----------------------------------------------------------------------------------------
%	ending
%----------------------------------------------------------------------------------------
\newpage
\begingroup
\thispagestyle{empty}
\AddToShipoutPicture*{\put(0,0){\includegraphics[scale=1.25]{esahubble}}} % Image background
\centering
\vspace*{5cm}
\par\normalfont\fontsize{35}{35}\sffamily\selectfont
\textbf{======Ending======}\\
{\LARGE Edited by 你的名字}\par % Book title
\vspace*{1cm}
{\Huge Thanks for reading\\ Wish you have a harvest}\par % Author name
\endgroup


\end{document}

